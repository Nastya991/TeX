\documentclass{article}

\usepackage[english,russian]{babel}
\usepackage[utf8x]{inputenc}
\usepackage{amsmath}

\begin{document}

\begin{center}
\textbf{МАТЕМАТИЧЕСКИЙ ТРИВИУМ}\\
\textrm{В.И.Арнольд}
\end{center}

\textrm{Уровень математической культуры падает; и студенты, и аспиранты выпускаемые нашими вузами, включая механико--математический факультет МГУ, становится не менее невежественными, чем профессора и преподаватели. В чем причина этого ненормального явления?В нормальных условиях студенты и аспиранты знают свою науку лучше профессоров, в соответствии с общим принципом распространения знаний: новое побеждает не потому, что старики его выучивают, а потому что приходят новые поколения, которые его не знают.\\
Среди множества причин,вызвавших это ненормальное положение, я хочу выделить те, которые зависят от нас самих,чтобы попытаться исправить то, что в наших силах. Одна из таких причин по -- моему,наша система экзаменов,специально рассчитанная на систематический выпуск брака,т.е. псевдоученых, которые математику выучивают как марксизм: зубря наизусть формулировки и ответы на наиболее часто встречающиеся экзаменационные вопросы.\\
Чем определяется уровень подготовки математика? Ни перечень курсов, ни их программы уровень не определяют.Единственный способ зафиксировать, чему мы действительно научили своих студентов-это перечислить задачи, которые они должны уметь решать в ходе обучения.
Речь идет не о каких-либо трудных задачах, а о тех простых вопросах, которые составляют строгий необходимый минимум. Этих задач не обязательно должны быть много, но уметь решать их нужно требовать жестко. И.Е.Тамм рассказывал, что, когда он попал во время гражданской войны к махновцам, во время допроса он сказал, что учился на физико-математическом факультете.И он остался жив лишь благодаря тому, что сумел решить задачу из теории рядов, которая была ему тут предложена, чтобы проверить правду ли он говорит. Наши студенты должны быть готовы к таким испытаниям!\\
Во всем мире экзамен по математике-это письменное решение задач. Письменный характер испытаний считается по всюду столь же обязательным признаком демократического общества,как выборы из нескольких кандидатов. Действительно, на устном экзамене студент полностью беззащитен.Мне случалось слышать, принимая экзамен на кафедре дифференциальных уравнений механико -- математического факультета МГУ, экзаменаторов, которые топили за соседнем столом студентов, дававших безукоризненные ответы(возможно, превосходящие уровень понимания преподавателя).Известны и такие случаи, когда топили нарочно(иногда от этого можно спасти,вовремя войдя в аудиторию).
Письменная работа --- это документ, и экзаменатор поневоле более объективный при ее проверке(особенно, если, как это и должно бы быть,работа для проверяющего анонимно).\\
Есть еще одно не маловажное преимущество письменных экзаменов:задачи остаются и могут быть опубликованы или сообщены студентам следующего курса для подготовки к своему экзамену. Кроме того, эти задачи сразу фиксируются и уровень курса, и уровень лектора, который их составил. Его сильные и слабые места сразу видны,специалисты сразу могут оценить преподавателя по тому, чему он хотел научить студентов и чему сумел научить их.\\
Между прочим, во Франции задачи общего для всей страны Concours general
(примерно соответствующего нашей олимпиаде) составляются учителями, посылающими свои задачи в Париж,где из них отбираются лучшие - и министерство получает объективные данные об уровне своих учителей, сравнивая, во-первых,предложенные ими задачи,а,во-вторых, результат их учеников. У нас же преподаватели оцениваются, как известно,по таким признакам как внешний вид,быстрота речи и идеологическая \glqq правильность\grqq.\\
Неудивительно, что наши дипломы не хотят признавать(думаю, что в дальнейшем это распространится и на диплом по математике).Оценки полученный на не оставляющих следов устных экзаменах, имеют не поддающейся объективному сравнению с чем бы то ни было,крайне расплывчатый и относительный вес,целиком зависящий от реального уровня преподавания и требования в данном вузе.При одних и тех же программах и отметках знаний и умения дипломников могут отличатся(в понятном смысле) в десятки раз. К тому же устный экзамен куда легче фальсифицировать(что случается и у нас, на механико-математическом факультете МГУ,где, как некогда сказал один незрячий преподаватель,приходится ставить хорошую отметку студенту \glqq отвечающему очень близко к учебнику\grqq, который не может ответить ни на одни вопрос.)\\
Сущность и недостатки нашей системы математического образования прекрасно описал Р.Фейман в своих воспоминаниях(\glqq Вы, конечно, шутите, мистер Фейман\grqq - глава о преподавании физики в Бразилии, русский перевод который опубликован в Успехах физических наук, т.148, в.3,1986).
ПО словам Феймана, студенты эти ничего не понимают,но никогда не задают вопросы,делая вид, что понимают все, А если кто-нибудь начинает задавать вопросы, то курс быстро ставит его на место,как зря отнимающего время у диктующего лекцию преподавателя и у записывающих ее студентов. В результате никто не может ничего из выученного применить ни в одном примере.Экзамен же(догматические, вроде наших:сформулируйте определение, сформулируйте теорему) благополучно сдаются.Студенты приходят в состояние \glqq самораспространяющейся псевдообразованности\grqq  и могут в дальнейшем подобным образом учить следующие поколения.Но вся эта деятельность полностью бессмысленна и фактически наши выпускники специалистов в значительной мере являются обманом, липой и приписками:эти так называемые специалисты не в состоянии решить простейших задач, не владеют элементами своего ремесла.\\
Итак \emph{чтобы положить конец припискам, нужно зафиксировать не список теорем, а набор задач, которые должны уметь решать студенты.Эти списки задач нужно ежегодно публиковать(думаю, список должен содержать задач по десять для каждого семестрового курса)} тогда мы увидим, чему мы реально учим студентов и насколько это удается. А для того, чтобы студенты научились применять свою науку,\emph{все экзамены нужно проводить только письменно}\\
Естественно, задачи от вуза к вузу и от года к году будут меняться.Тогда можно будет сравнивать уровень разных преподавателей и выпускников разных лет.Студент, которому для вычисления с десятипроцентной точностью среднего от сотой ступени синуса требуется значительно больше пяти минут, не владеет математикой,даже если он занимался нестандартным анализом, универсальными алгебрами, супермногообразиями или теоремами вложения.\\
Составление эталонных задач --- трудоемкая работа, но я думаю, ее необходимо проделать. В качестве попытки я предлагаю нижний список из ста задач как математический минимум студента--физика. Эталонные задачи(в отличии от программ) не определены однозначно, и многие, вероятно, со мной не согласятся. Тем не менее я считаю, что начать фиксировать уровень математических требований при помощи письменных экзаменов и эталонных задач необходимо. Хочется надеяться, что в будущем студенты будут получать эталонный задачи по каждому курсу в начале каждого семестра,а начетническизубрилбные устные экзамены уйдут в прошлое.}\\
\begin{enumerate}
\item Нарисовать график производной и график интеграла функции, заданной свободно начерченным графиком.
\item Найти предел: $$\lim_{x\to 0}\frac{\sin\tan x-\tan\sin x}{(\arcsin\arctan x-\arctan\arcsin x)}$$
\item Найти критические значения и критические точки отображения $z\mapsto x^{2}-2\bar{z}$ нарисовать ответ
\item Вычислить сотую производную  $$\frac{(x^{2}+1)}{x^{3}-x}$$
\item Вычислить сотою производную $\frac{1}{x^{2}+3x+2}$
 в нуле с относительной погрешностью 10\%
\item Нарисовать на плоскости (x,y) кривую, заданную параметрически: $$x=2t- 4t^{3}, y=t^{2}-3 t^{4}$$
\item Сколько нормалей к эллипсу можно провести из данной точки плоскости? Исследовать область, в которой число нормалей максимально.
\item Сколько максимумов,минимумов и седел имеет $x^{4}+y^{4}+z^{4}+u^{4}+\upsilon^{4}$ на поверхности $$x+...+\upsilon=0,x^{2}+...+\upsilon^{2}=1,
x^{3}+...+\upsilon^{3}=C$$
\item Всякий ли положительный многочлен от двух вещественных переменных достигает своей нижней грани на плоскости?
\item Исследовать асимптотики решений y уравнения $x^{5}+x^{2}y^{2}=y^{6}$? стремящихся к 0 при $x\mapsto0$.
\item Исследовать сходимость интеграла $$\iint_{-\infty}^{+\infty}\frac{dxdy}{1+x^{4}\dot{x^{4}}}$$
\item Найти поток векторного поля $\vec{r}\setminus r^{3}$ через поверхность $$(x-1)^{2}+y^{2}+z^{2}=2$$
\item Вычислить с относительно погрешностью 5\% $$\int_0^11x^{x}dx$$
\item Вычислить с относительной погрешностью не более 10\% $$\int_{-\infty}^{+\infty}(x^{4}+4(x+4))^{-100}dx.$$
\item Вычислить с относительной погрешностью 10\% $$\int_{-\infty}^{+\infty}\cos(100(x^{4}-x))dx.$$
\item Какую долю от объема пятимерного куба составляет объем вписанного в него шара? А от десятимерного?
\item Найти расстояние от центра тяжести однородного 100-мерного полушара радиуса 1 до центра шара с относительной погрешностью 10\%.
\item Вычислить $$\int...\int e^{-\sum_{1\ll i\ll j\ll n}x_{i} x_{j}}dx_{1}\dots dx_{n}$$ 
\item Исследовать ход лучей в плоской среде с показателем преломления $n(y)=y^{4}-y^{2}+1,$ пользуясь законом Спеллиуса $n(y)\sin(\alpha)=const$, где $\alpha$ - угол луча с осью y.
\item Найти производную решения уравнение $\ddot{x}=x+A\dot{x}^{2}$ с начальным условием x(0)=1, $\dot{x}(0)=0$, по параметру А при $А=0$.
\item Найти производную решения $\ddot{x}=\dot{x}^{2}+x^{3}$ с начальным условием \\ $x(0)=0,\dot{x}(0)=A$ по А при А=0.
\item Исследовать границу области устойчивости $(max Re\lambda_{i}<0)$ в пространстве коэффициентов уравнения $$\dddot{x}+a\ddot{x}+b\dot{x}+cx=0.$$
\item Решить квазиоднородное уравнение  $\frac{\partial{y}}{\partial{x}}=\frac{x+x^{3}}{y}$
\item Решить квазиоднородное уравнение $\ddot{x}=x^{5}+x^{2}\dot{x^{2}}$
\item Может ли асимптотически устойчивое положение равновесия сделаться не устойчивым по Ляпунову по линеаризации?
\item Исследовать поведение при $t\mapsto +\infty$ решений системы 
$$\begin{cases}                             
\dot{x} &= y, \\                       
\dot{y} &= 2\sin y-y-x,               
\end{cases}  \quad                        
\begin{cases}
\dot{x} &= y, \\
\dot{y} &= 2x-x^{3}-x^{2}-\varepsilon y,
\end{cases}$$
где $\varepsilon\ll1.$
\item Нарисовать образы решения уравнения $$\ddot{x}=F(x)-\kappa\dot{x}, F(x)=-\frac{dU}{dx}$$ на плоскости (x,E), где $E=\dot{x}^{2}/2+U(x)$ вблизи невырожденных критических точек потенциала.
\item Нарисовать фазовый портрет и исследовать его изменения при изменении малого комплексного параметра $\varepsilon$: $$\dot{z}=\varepsilon z-(1+i)z|z|^{2}+\bar{z}^{4}$$
\item Заряд движется со скоростью 1 по плоскости под действием перпендикулярного ей сильного магнитного поля B(x,y). В какую сторону будет дрейфовать центр ларморовской окружности? Выслать скорость этого дрейфа(в первом приближении).[Математически речь идет о кривых кривизны $N\mapsto\infty$].
\item Найти сумму индексов особых точек векторного поля $z\bar{z}^{2}+z^{4}+2\bar{z}^{4}$, отличных от нуля.
\item Найти индекс особой точки 0 векторного поля с компонентами $$(x^{4}+y^{4}+z^{4}, x^{3}y-xy^{3}, xyz^{2})$$
\item Найти индекс особой точки 0 векторного поля $$ grad(xy+yz+xz)$$
\item Найти коэффициент зацепления фазовых траекторий уравнения малых колебаний $\ddot{x}=-4x,\ddot{y}=-9y$ на поверхности уровня полной энергии.
\item Исследовать особые точки кривой $y=x^{3}$ на проективной плоскости.
\item Нарисовать геодезические на поверхности $$(x^{2}+y^{2}-2)^{2}+z^{2}=1$$
\item Нарисовать эвольвенты кубической параболы $y=x^{3}$ (эвольвента - это геометрическое место точек $\vec{r(s)}+(c-s)\dot{\vec{r(s)}}$, где s- длина вдоль кривой $\vec{r(s)}$, $с$ - константа).
\item Доказать что поверхность в евклидовом пространстве $$((A-\lambda E)^{-1}x,x)=1,$$ проходящие через точку x и соответствующие $\lambda$ (А - симметрический оператор без кратных собственных чисел) попарно ортогональны.
\item Вычислить интеграл от гауссовской кривизны поверхности $$z^{4}+(x^{2}+y^{2}-1)(2x^{2}+3y^{2}-1)=0.$$
\item Вычислить интеграл Гаусса $$\oint\oint\frac{d\vec{A}, d\vec{B},\vec{A}-\vec{B}}{|\vec{A}-\vec{B}|^{3}}$$ где $\vec{A}$ пробегает кривую $x=\cos\alpha, y=\sin\alpha, z=0$, а $\vec{B}$ - кривую $x=2\cos\beta, y=\frac{1}{2}\sin\beta, z=\sin2\beta$
\item Перенести параллельно направленный в Ленинград (широта 60$^{0}$) на север вектор с запада на восток вдоль замкнутой параллели.
\item Найти геодезическую кривизну прямой $y=1$ на верхней полуплоскости с метрикой Лобачевского-Пуанкаре $$ ds^{2}=(dx^{2}+dy^{2})/y^{2}$$
\item Пересекаются ли в одной точке медианы треугольника на плоскости Лобачевского? А высоты?
\item Найти числа Бетти поверхности $x_{1}^{2}+...+x_{k}^{2}-y_{1}^{2}- ...-y_{l}^{2}=1$ и множества$x_{1}^{2}+...+x_{k}^{2}\leq y_{1}^{2}- ...-y_{l}^{2}$ в $k+l$ - мерном линейном пространстве
\item Найти числа Бетти поверхности $ x^{2}+y^{y}=1+z^{2}$ в трехмерном проективном пространстве. Тоже для поверхности $z=xy,z=x^{2}, z=x^{2}+y^{2}.$
\item Найти индекс самопересечения поверхности $x^{4}+y^{4}=1$ в проективной плоскости СР$^{2}$.
\item Отобразить конформно внутренность единичного круга на первый квадрант.
\item Отобразить конформно внешность круга на внешность данного эллипса.
\item Отобразить конформно полуплоскость без перпендикулярного ее краю отрезка на полуплоскость.
\item Вычислить $$ \oint_{|z|=2} \frac{dz}{\sqrt1+z^{10}}$$
\item Вычислить $$ \int_{-\infty}^{+\infty}\frac{e^{ikx}}{1+x^{2}}dx$$
\item Вычислить интеграл $$ \int_{-\infty}^{+\infty}e^{ikx}\frac{1-e^{x}}{1+e^{x}}dx$$
\item Вычислить первый член асимптотики при $k\mapsto\infty$ интеграл $$\int_{-\infty}^{+\infty}\frac{e^{ikx}dx}{\sqrt1+x^{2n}}$$
\item Исследовать особые точки дифференциальной формы $dt=dx/y$ на компактной римановой поверхности $y^{2}/2+U(x)=E,$ где U-многочлен, а Е - критическое значение.
\item $\ddot{x}=3x-x^{3}-1.$ В которой из ям больше период колебаний(в более мелкой или более глубокой) при равных значениях полной энергии?
\item Исследовать топологически риманова поверхность функции $$\omega=\arctan z$$
\item Сколько ручек имеет риманова поверхность функции $$\omega=\sqrt{1+z^{n}}$$
\item Найти размерность пространства решений задачи $\partial{x}/\partial\bar{z}=\delta(z-i)$ при $Imz\geq 0,Imu(z)=0$ при $Imz=0, u\mapsto0$ при $ z\mapsto\infty$.
\item Найти размерность пространства решений задачи $\partial{x}/\partial\bar{z}=a\delta(z-i)+b\delta(z+i)$ при $|z|\leq2, Imz=0$ при |z|=2.
\item Исследовать существование и единственность решения задачи $yu_{x}=xu_{y}, u|_{x=1}=\cos y$ в окрестности точки (1,$y_{0}$).
\item Существует ли и единственно ли решение задачи Коши $$x(x^{2}+y^{2})\frac{\partial U}{\partial x}+y^{3}\frac{\partial U}{\partial y} =0,\quad U|_{y=0}=1$$ в окрестности точки (x$_{0}$,0) оси x?
\item При каком наибольшем t решение задачи $$\frac{\partial U}{\partial t}+U\frac{\partial U}{\partial x}=\sin{x}, U|_{t=0}=0$$ продолжается на интервале[0,t)?
\item Найти все решения уравнения $y\partial u/\partial x-\sin x\partial u/\partial y=u^{2}$ в окрестности точки (0,0).
\item Существует ли решение задачи Коши $y\partial u/\partial x-\sin x\partial u/\partial y=y, u|_{x=0}=y^{4}$ о всей плоскости (x,y)? Единственно ли оно?
\item Имеет ли задача Коши $u|_{y=x^{2}}=1,(\nabla u)^{2}=1$ гладкое решение в области $y\geq x^{2}?$ В области $y\leq x^{2}?$
\item Найти среднее значение функции $\ln r$ на окружности $(x-a)^{2}+(y-b)^{2}=R^{2}$ (функция 1/r на сфере).
\item Решить задачу Дирихле $$ \Delta u=0 \quad\text{при}\quad x^{2}+y^{2}<1;\\ u=1 \quad\text{при}\quad x^{2}+y^{2}=1,\quad y>0;\\ u=-1 x^{2}+y^{2}=1, \quad y<0$$
\item Какова размерность пространства непрерывных при $x^{2}+y{2}\leq1$ решение задачи  $$\Delta u=0 \quad\text{при}\quad x^{2}+y{2}>1,\quad \partial u/\partial n=0 \quad\text{при}\quad x^{2}+y^{2}=1?$$
\item Найти $$ inf \int\int x^{2}+y^{2}\leq1 \frac({\partial u}{\partial x})^{2}+\frac({\partial u}{\partial y})^{2}dxdy$$ по $С^{\infty}$ - функциям $u$, равным 0 в 0 и 1 при $x^{2}+y^{2}=1$
\item Доказать, что телесный угол,опирающийся на задний замкнутый контур, - гармоническая вне контура функция вершины угла.
\item Вычислить среднее значение телесного угла, под которым видно круг $x^{2}+y^{2}\leq1,$ лежащий в плоскости z=0, из точек сферы $$x^{2}+y^{2}+(z-2)^{2}=1.$$
\item Вычислить плотность заряда на проводящей границе плоскости $x^{2}+y^{2}+z^{2}=1,$ в которую помещен заряд равный q=1 на расстоянии r от центра.
\item Вычислить в первом приближении по $\varepsilon$ влияние сжатия Земли $(\varepsilon\approx1/300)$ на гравитационное поле Земли на расстоянии Луны(считая Землю однородной)
\item Найти(в первом приближении по $\varepsilon$) влияние несовершенства почти сферического конденсатора $R=1+\varepsilon f(\phi,\theta)$ на его емкости.
\item Нарисовать график u(x,1), если $0\leq x\leq1$, $$ \frac{\partial u}{\partial t}=\frac{\partial^{2}u}{\partial x^{2}}, \quad u|_{t=0}=x^{2},\quad u|_{x^{2}=x}=x^{2}.$$
\item Вследствие годовых колебаний температуры Земля в городе N промерзает на глубину 2м. На какую глубину она промерзла бы, вследствие суточных колебаний такой же амплитуды?
\item Исследовать поведение $t\mapsto+\infty$ решения задачи $$u_{t}+(u\sin x)_{x}=\varepsilon u_{x{x}}, u|_{t=0}\equiv1,\varepsilon\ll1$$
\item Найти собственный числа оператора Лапласа $\Delta=\operatorname{div}grad$ на сфере радиуса R в евклидовом пространстве размерности n и их кратности.
\item Решить задачу Коши $$\frac{\partial^{2}A}{\partial t^{2}}=9\frac{\partial^{2}A}{\partial x^{2}}-2B, \frac{\partial^{2}B}{\partial t^{2}}=6\frac{\partial^{2}B}{\partial x^{2}}-2A,$$
$$ A|_{t=0}=\cos x,\quad B|_{t=0}=0, \quad\frac{\partial A}{\partial t}|_{t=0}=\frac{\partial B}{\partial t}|_{t=0}=0.$$
\item Сколько решений имеет краевая задача $$u_{x{x}}+\lambda u=\sin x, u(0)=u(\pi)=0?$$
\item Решить уравнение $$ \int_0^1(x+y)^{2}u(x)dx=\lambda u(y)+1.$$
\item Найти функцию Грина оператора $d^{2}/dx^{2}-1$ решить уравнение $$\int_{-\infty}^{+\infty}e^{-|x-y|}u(y)dy=e^{-x^{2}}. $$
\item При каких значениях скорости с уравнение $u_{t}=u-u^{2}+U_{x{x}}$ имеет решение в виде бегущей волны $u=\varphi(x-ct),\varphi(-\infty)=1,\varphi(\infty)=0,0\leq u\leq1?$
\item Найти решения уравнения $u_{t}=u_{x_{x_{x}}}+uu_{x},$ имеющие вид бегущей волны $u=\varphi(x-ct),\varphi(\pm\infty)=0.$
\item Найти число положительных и отрицательных квадратов в нормальной форме квадратичной формы $\sum_{i<j}(x_{i}-x_{j})$ от n переменных. А для формы $\sum_{i<j}x_{i}x_{j}?$
\item Найти длины главных осей эллипсоида $$\sum_{i\leq j}x_{i}x_{j}=1$$
\item Через центр куба (тетраэдра,икосаэдра) провести прямую так, что сумма квадратов расстояний до вершин была
\begin{enumerate}
\item минимальной;
\item максимальной.
 \end{enumerate}
\item Найти производные длин полуосей эллипсоида $x^{2}+y^{2}+z{2}+xy+yz+xz=1+\varepsilon xy$ по $\varepsilon$ при $\varepsilon=0$.
\item Какие фигуры могут получится при пересечении бесконечномерного кубы $|x_{k}|\leq1, k=1,2...,$ двумерной плоскостью?
\item Вычислить сумму векторных произведений $[[x,y]z]+[[y,z]x]+[[z,x]y].$
\item Вычислить сумму коммутаторов матриц, $[A[B,C]]+[B[C,A]]+[C[A,B]]$ где $[A,B]=AB-BA.$
\item Найти жорданову нормальную форму оператора $e^{d/dt}$ в пространстве квазимногочленов ${e^{\lambda t}p(t)}$, где степени многочленов p меньше 5, оператор $ad_{A}, B\mapsto[A,B]$ в пространстве $(n\times n)$-матриц $B$, где $А$--диагональная матрица.
\item Найти порядок подгруппы группы вращений куба и ее нормальные делители.
\item Разложите пространство функций, заданных в вершинах куба, на инвариантные подпространства, неприводимые относительные группы 

\begin{enumerate}
\item симметрии 
\item его вращений.
\end{enumerate}
\item Разложите пятимерное вещественное линейное пространство на неприводимые инвариантные подпространства группы, порожденные циклической перестановкой базисных векторов.
\item Разложить пространство однородных многочленов пятой степени от (x,y,z) на неприводимые подпространства, инвариантные относительно группы вращений SO(3).
\item Каждый их 3600 абонентов телефонной станции вызывает ее в среднем раз в час.Какова вероятность того, что в данную секунду поступит 5 или более вызовов? Оценить средний промежуток времени между такими секундами (i,j,1).
\item Частица, блуждающая по целым точкам полуоси $x\geq0,$ с вероятностью a сдвигается на 1 в право, с вероятность b влево, в остальных случаях остается на месте( при $x=0$ вместо сдвига на лево остается на месте). Определить установившееся распределение вероятностей, а также математическое ожидание x и математическое ожидание $x^{2}$ через большое время, если в начале частица находилась в точке 0.
\item Каждый участник игры в очко на пальцах, стоящих по кругу, выбрасывает несколько пальцев правой руки,после чего для определения победителя суммарное число выкинутых пальцев отсчитывается по кругу от водящего. При каком числе участников N вероятность выигрыша хотя бы одного из подходящих N/10 участников становится больше 0,9?Как ведет себя $N\mapsto\infty$ вероятность выигрыша водящего?
\item Один из игроков прячет монету в 10 или 20 копеек, а другой отгадывает. Отгадавший получает монету, не отгадавший платит 15 копеек. Честная ли это игра?Какие оптимальные смешанные стратегии обоих участников?
\item Найти математическое ожидание площади проекции куба с ребром 1 на плоскости при изотропно распределенным случайном направлении проектирования.
\end{enumerate}
\end{document} 